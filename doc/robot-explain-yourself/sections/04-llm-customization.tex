\section{LLM Customization}

\subsection{Model selection and comparison}

The project evaluated multiple LLM architectures:

\begin{table}[ht]
\centering
\begin{tabular}{|l|l|l|}
\hline
\textbf{Model} & \textbf{Size} & \textbf{Performance} \\
\hline
Llama 3.2           & 2.0 GB    & Baseline performance \\
nous-hermes2:latest & 6.1 GB    & small model that explain lightly \\
deepseek            & 8.1 GB    & Good reasoning, efficient and clear\\
qwen3:30b-a3b       & 18 GB     & big model that provides good answers but take some time \\
\hline
\end{tabular}
\caption{LLM model comparison}
\end{table}

\subsection{Prompt engineering}

Effective prompt design proved crucial for generating relevant explanations. The system uses structured prompts that include:
\begin{itemize}
    \item Environmental context and sensor readings
    \item Historical path information
    \item User questions and interaction history
    \item Domain-specific constraints and objectives
    \item Additional datas provided by the human or captors
\end{itemize}

\subsection{Fine-tuning Approach}

The customization process involved:
\begin{enumerate}
    \item Dataset creation with robot-specific scenarios
    \item Prompt template
    \item Response quality evaluation and iteration
    \item Integration with robot perception systems (In the future)
\end{enumerate}

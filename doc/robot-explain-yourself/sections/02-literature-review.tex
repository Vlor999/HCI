\section{Literature review}

\subsection{Technologies and methodologies in Human-Robot Communication}

\paragraph{Dialogue management in HRI :}
The current state of dialogue management in human-robot interaction is reviewed in survey of dialogue management in human-robot interaction~\cite{dialogue-hri-review}.
They evaluate capabilities, methods, and challenges, and emphasize the need for structured approaches that effectively combine HRI with dialogue systems.
They show how to discuss properly with a robot powered by Artificial Intelligence but also why it's important to understand the interactions.

\paragraph{Human–robot interaction :}
An overview of human–robot interaction is provided in wikipedia : Human-robot interaction~\cite{hri-overview}.
The idea behind this article is to show that to interact we need to create a safe and intuitive interaction : reducing friction. To be specialize into a specific area is the key point to create this safe zone of communication.
Involving also feelings, emotions and gesture makes the conversartions more real and push the user to feel confortable and then to continue this conversation/interactions.

\subsection{Interpretability of low-level perception in robotics}

\paragraph{Explainable robotic systems in scenarios :}
Puiutta and Veith~\cite{explainable-rl} discuss explainability in robotic reinforcement learning agents.
Explainable robotic systems are crucial for human-robot collaboration.
This study compares three approaches to computing the probability of success for reinforcement learning agents in robotic scenarios, finding learning-based and introspection-based approaches to be suitable alternatives to a memory-based baseline.

\paragraph{Subsumption architecture :}
The subsumption architecture~\cite{subsumption} is a reactive robotic architecture that decomposes behavior into sub-behaviors organized into a hierarchy of layers.
Each layer implements a level of behavioral competence, with higher layers subsuming lower ones to create viable behavior.
This architecture emphasizes iterative development, task-specific perception, and parallel control, enabling real-time interaction with dynamic environments.

\subsection{Challenges and opportunities for understandable robotics}

\paragraph{Trust in explainable robots :}
Trust calibration and explanation-specificity in robots are discussed in Wang et al.~\cite{trust-explainable}, who offer insight into designing systems that feel transparent and reliable.
Explainable AI (XAI) and explainable robots literature aims to enhance human understanding and human-robot team performance.
The paper discusses three trust-related considerations for explainable robot systems: bases of trust, trust calibration, and trust specificity.

\paragraph{Effects of robot explanations :}
The authors of~\cite{explanations-effect} further evaluate the impact of robot explanations on user perception.
Their findings indicate that explanations make robots appear more lively and human-like, leading to more interactive conversations and an increased likelihood of users believing the robot.

\paragraph{Ethical black box :}
Professor Marina Jirotka’s concept of the “ethical black box”~\cite{ethical-black-box} suggests embedding inflight recorders in robots to promote accountability and auditability of robotic decisions.

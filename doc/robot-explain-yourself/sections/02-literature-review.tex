\section{Literature Review}

This literature review summarizes the current state of technologies and methodologies in human-robot communication, focusing particularly on interpretability and explainability of low-level perception information, and the associated challenges and opportunities in making robotic behavior understandable to human users.

\subsection{Human-Robot Communication Technologies}

Current approaches to human-robot communication span multiple modalities including speech, gesture, and visual interfaces.
Recent advances in dialogue management systems have shown promise in creating more natural interactions \cite{dialogue_management_2023}.

\subsection{Explainable AI in Robotics}

The field of explainable AI (XAI) has gained significant attention, particularly in safety-critical applications.
Trust calibration and explanation specificity have emerged as key factors in building reliable human-robot relationships \cite{trust_explainable_robots_2020}.

\subsection{Large Language Models in Robotics}

Recent work has explored the integration of LLMs with robotic systems for various tasks including instruction following, task planning, and human-robot dialogue.
The emergence of foundation models has opened new possibilities for natural language interfaces in robotics \cite{llm_robotics_2024}.

\subsection{Challenges and Opportunities}

Key challenges include real-time processing constraints, the interpretability gap between sensor data and natural language, and the need for context-aware explanations.
Opportunities exist in leveraging pre-trained language models and developing domain-specific fine-tuning approaches.

\section{Literature review}

\subsection{Technologies and methodologies in Human-Robot Communication}

\paragraph{Dialogue management in HRI}
The current state of dialogue management in human-robot interaction is reviewed by Lopes et al.~\cite{dialogue-hri-review}. They evaluate capabilities, methods, and challenges, and emphasize the need for structured approaches that effectively combine HRI with dialogue systems.

\paragraph{Human–robot interaction}
An overview of human–robot interaction is provided in a community-maintained source~\cite{hri-overview}. It emphasizes intuitive communication through speech, gestures, and facial expressions, and underlines the importance of designing robots that “feel” human.

\subsection{Interpretability of low-level perception in robotics}

\paragraph{Explainable robotic systems in scenarios}
Puiutta and Veith~\cite{explainable-rl} discuss explainability in robotic reinforcement learning agents. They present methods for making decision probabilities interpretable to end-users.

\paragraph{Subsumption architecture}
The subsumption architecture~\cite{subsumption} is an early robotics design that focuses on real-time responsiveness and intelligent behavior emerging from layered, embodied interactions.

\subsection{Challenges and opportunities for understandable robotics}

\paragraph{Trust in explainable robots}
Trust calibration and explanation-specificity in robots are discussed in Wang et al.~\cite{trust-explainable}, who offer insight into designing systems that feel transparent and reliable.

\paragraph{Effects of robot explanations}
The impact of robot explanations on user perception is further evaluated by the same authors~\cite{explanations-effect}. Their findings suggest explanations make robots seem livelier and more human-like.

\paragraph{Ethical black box}
Professor Marina Jirotka’s concept of the “ethical black box”~\cite{ethical-black-box} suggests embedding inflight recorders in robots to promote accountability and auditability of robotic decisions.

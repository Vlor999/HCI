\section{Evaluation}

\subsection{Evaluation methodology}

The system evaluation employed multiple approaches:
\begin{itemize}
    \item \textbf{Automated testing}: Unit tests for core functionality and edge cases. Run when you commit something
    \item \textbf{Explanation quality assessment}: Keyword matching and semantic similarity analysis
    \item \textbf{User study design}: Interactive scenarios with human participants
    \item \textbf{Performance metrics}: Response time, accuracy, and user satisfaction measures (not really efficient but look if the answer makes sense)
\end{itemize}

\subsection{Quantitative results}

The evaluation framework assessed explanations based on:
\begin{itemize}
    \item Keyword coverage score (target: $\geq$ 75\%)
    \item Factual accuracy in decision reasoning
    \item Consistency across similar scenarios
\end{itemize}

\subsection{Qualitative findings}

Key observations from the evaluation include:
\begin{itemize}
    \item Users appreciated natural language explanations
    \item Context-aware responses significantly improved user understanding
    \item Explanation quality varied with environmental complexity
    \item Interactive questioning enhanced user engagement and trust
\end{itemize}

\subsection{Limitations and challenges}

Identified limitations include:
\begin{itemize}
    \item Computational overhead for real-time processing
    \item Occasional over-fitting in response patterns
    \item Context window limitations for extended conversations
    \item Need for domain-specific fine-tuning for optimal performance
\end{itemize}

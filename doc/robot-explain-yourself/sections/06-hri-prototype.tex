\section{Human-Robot Interaction Prototype}

\subsection{System implementation}

The prototype system was implemented using \textbf{Python} with the following key components:
\begin{itemize}
    \item \textbf{Path processing module :} Handles environmental data and context extraction.
    \item \textbf{LLM interface :} Manages communication with local Ollama server.
    \item \textbf{Conversation manager :} Maintains interaction history and context.
    \item \textbf{User interface :} Provides command-line and potential interaction.
    \item \textbf{Storing conversations :} Store all the previous conversations but also the new datas provided during the conversation.
\end{itemize}

\subsection{Core features}

The implemented system supports:
\begin{itemize}
    \item Interactive questioning about robot path decisions.
    \item Real-time explanation generation.
    \item Context-aware responses based on environmental conditions.
    \item Conversation logging and history management.
    \item Multiple scenario support for testing and evaluation.
\end{itemize}

\subsection{Technical architecture}

The system architecture follows a modular design:
\begin{itemize}
    \item \texttt{src/robotPathExplanation.py}: Main application logic
    \item \texttt{src/core/path.py}: Data structures for path and environmental information
    \item \texttt{src/llm/llmModel.py}: LLM integration and prompt management
    \item \texttt{src/logging/conversationLogger.py}: Interaction recording and analysis
\end{itemize}

\subsection{Usage scenarios}

The prototype supports various interaction scenarios:
\begin{itemize}
    \item Path selection queries (e.g., "Which path should I take if I want the easiest route?")
    \item Safety-related questions (e.g., "I have a heavy load. Which path is safest?")
    \item Time-constrained decisions (e.g., "I am in a hurry but want to avoid danger")
    \item Real-time updated decision (e.g., "Which path should i took according to the new datas that I provided ?")
\end{itemize}

\section{Different phases of the project}

\subsection{Phases of the project}
The project started in March 2025 and as been divided into several phases.
The main phases are:
\begin{itemize}
    \item \textbf{Phase 1: Literature review} - Research existing HRI frameworks and LLM applications.
    \item \textbf{Phase 2: Framework design} - Develop a modular architecture for the HRI system.
    \item \textbf{Phase 3: LLM customization} - Fine-tune LLMs for specific HRI tasks.
    \item \textbf{Phase 4: Evaluation} - Test and validate the system with real users.
    \item \textbf{Phase 5: Documentation and reporting} - Document the framework, results, and lessons learned.
\end{itemize}

\subsection{Project timeline}
So firstly to understand what have already been done and what is needed was the main goal of the first phase.
So I had to do a literature review of the current state of the art in HRI and LLMs but more generaly to understand how does work the local LLMs.
This part took around 2-4 weeks but in fact it never ended because I always find new papers that can be useful for the project.
Moreover, you always found new informations or new points that can make your LLM more useful but also more efficient.
As a second phase I had to design the framework that will be used to answer the user questions.
This part was quite easy but took some time (4-6 weeks)since we still have to find the most efficient way to do it and to choose the language to use, how to structure everything and how to make it easy to use.
The third phase was to customize the LLM to make it more efficient and to make it able to answer the user questions.
Next, the evaluation phase focused on optimizing the framework by rewriting parts of the code for improved efficiency and usability, as well as developing better prompts for the LLM.
this previous phase took around 2-3 weeks.
Finally, the documentation and reporting phase involved writing this report and preparing the final presentation, which took about 1-2 weeks.
